\newglossaryentry{amont}
{
    name=amont,
    description={Un évènement situé en amont d'un autre se déroule avant ce dernier}
}
\newglossaryentry{distance}
{
    name=distance,
    description={On parlera de distance pour décrire une durée entre deux dates}
}
\newglossaryentry{relationTemporelle}
{
  name={relation temporelle},
  description={Une relation temporelle décrit un intervalle de temps, plus ou moins rigide, séparant deux instants.},
  %first={\glsentrydesc{relationTemporelle} (\glsentrytext{relationTemporelle})},
  plural={relations temporelles}
  %descriptionplural={},
  %firstplural={\glsentrydescplural{relationTemporelle} (\glsentryplural{relationTemporelle})}
}
\newglossaryentry{timenode}
{
  name={timenode},
  plural={timenodes},
  description={Un timenode, ou noeud temporel, représente un instant. On s'en sert notamment pour synchroniser des évènements}
}

\newglossaryentry{patch}
{
  name={patch},
  plural={patchs},
  description={Un patch est }
}

%usages
%\gls{}
%\glspl{}
%\Gls{}
%\Glspl{}

% acronymes
\newacronym{csp}{CSP}{Constraint Satisfaction Problem}
\newacronym{tdd}{TDD}{Test Driven Developpment}
\newacronym{scrime}{\scrime{}}{Studio de Création et de Recherche en Informatique et Musique Electroacoustique}
\newacronym{osc}{\osc{}}{Open Sound Control}

%\newacronym{clef}{display}{explication}
%\newacronym{}{}{}

%usage : 
%\acrshort{}
%\acrlong{}
%\acrfull{}




