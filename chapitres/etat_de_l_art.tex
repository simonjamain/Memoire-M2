\chapter{État de l'art}
Dans ce chapitre nous introduisons le concept de programmation par contrainte et nous étudions les solutions existantes. Nous regardons également les applications qui mettent en oeuvre ces principes et présentons le logiciel \iscore{}.
\section{la programmation par contrainte}
\begin{citeauteur}[Olivier Delerue\cite{Delerue2003openmusic}]
Une contrainte est une relation, au sens mathématique que l’on souhaite à tout instant vérifiée. La programmation par contrainte consiste à aborder un problème au moyen de contraintes, d’une part en le décrivant complètement sous forme d’un ensemble de ces relations et d’autre part en produisant des algorithmes permettant d’aboutir rapidement à une solution, c'est-à-dire à un élément qui précisément vérifie toutes les propriétés énoncées.
\end{citeauteur}

La programmation par contraintes est une manière très déclarative d'aborder un problème, la première étape est de bien définir quel est le problème à résoudre, puis de le traduire sous forme de contraintes -- ce que nous appelons le modèle -- , ensuite, il s'agit d'utiliser un système (solveur de contrainte) qui nous permet de résoudre ces contraintes pour obtenir une solution le cas échéant.

\begin{citeauteur}[Eugene C. Freuder\cite{paper:freuder:1997}]
Constraint Programming represents one of the closest approaches computer science has yet made to the Holy Grail of programming: the user states the problem, the computer solves it.
\end{citeauteur}

\subsection{Utilisation et role d'un solveur de contrainte}

Un solveur de contraintes accepte des variables de domaines définit (liste des valeurs possibles) ainsi que des contraintes sur leurs valeurs, et retourne une ou plusieurs solutions \figureRef{fonctionnementSolveur}. Nous verrons que dans le cas ou il existe plusieurs solutions, il est possible de définir des critère pour trier les solutions.

\schema{fonctionnementSolveur}{Un solveur de contraintes accepte des variables ainsi que des contraintes par rapport à celles-ci et tente de fournir une solution, c-à-d des valeurs de variables qui satisfassent toutes les contraintes posées.}

Je vais illustrer l'utilisation d'un solveur de contrainte par une version simple d'un exemple classique utilisé notamment dans la documentation\footnote{Manuel d'utilisateur \ortools{} \href{https://drive.google.com/file/d/0B7kalaGEbfufQ0h0N3J4TkVVaDQ/view?usp=sharing}{or-tools manual}, section 2.3, exemple CP+IS+FUN=TRUE}\footnote{\href{http://www.gecode.org/doc-latest/MPG.pdf}{Manuel d'utilisateur \gecode{}, chapitre 2, exemple SEND+MORE=MONEY}} de deux implémentations que je présenterais plus tard.

Cet exemple et celui du puzzle "cryptarithméique" le problème et le suivant :

Compte tenu la phrase suivante BO + NJ = OUR existe t'il un moyen de remplacer chaque lettre par un chiffre\footnote{On prendra des chiffres en base 10 mais on pourrait également changer la base.} pour que l'équation soit correcte? On précise que la valeur assignée à des lettres identiques doit être la même, on ajoutera également que la valeur de lettres différentes doit être également différente et que chaque nombre formé par un mot ne doit pas commencer par un zéro.

Pour obtenir une solution nous déclarons les différentes variables entières ainsi que leur domaine qui est donc $[0,1...9]$ pour J, U et R et $[1,2...9]$ pour B, N et O car ils se trouvent en début de mot.

\emph{À noter que nous aurions pu utiliser des domaines plus vaste et les restreindre grâce à des contraintes, mais c'est en général plus coûteux et moins explicite pour cet exemple.}

Nous allons maintenant établir les contraintes nécessaires, et nous allons voir que ces contraintes découlent naturellement des spécifications du problème et qu'elles peuvent s'exprimer sous forme d'équation.

Nous voulons les valeurs de nos varibles B,O,N,J,U et R toutes différentes pour cela nous pouvons l'exprimer par de multiples contraintes de type $B \neq O$ etc... Pour nous aider, les solveurs de contraintes proposent en général des contraintes de plus haut niveau, et c'est le cas de la contrainte \textbf{allDifferent} qui réalise précisément ce que l'on veut, nous entrons donc la contrainte $\mathrm{allDifferent}([B,O,N,J,U,R])$.

Il ne nous reste plus qu'a exprimer la relation entre nos mots, pour ceci, nous pouvons décomposer en définissant $mot1$, $mot2$ et $mot3$ puis en posant $mot1+mot2=mot3$ ou directement en une ligne telle que : $(B*10 + O) +(N*10 +J) = O*100+U*10+R$.

Le solveur a maintenant toutes les données pour trouver des solutions à notre problème, vous trouverez l'implémentaiton de ce probleme en annexe \ref{annexePuzzle}, pour information une des solutions est ${B:2,O:1,N:8,J:3,U:0,R:4}$ soit $21+83=104$.




\subsection{les solveurs de contraintes}
Dans le cadre de ce stage, il n'est pas question de créer solveur de contraintes, c'est une tâche très complexe et des implémentations existent déjà, nous devrons cependant faire un choix avisé concernant l'implémentation à utiliser.

Tout d'abord, il est important de savoir que le logiciel \iscore{} est codé en langage \cpp{}, nous allons donc préférer des solveurs proposant une interface dans ce langage pour une utilisation finale. À noter que la liste suivante n'est pas exhaustive.%finir phrase mieux

\subsubsection{\matlab{}}
Un des outils que j'aime utiliser pour faire du prototypage est \matlab{}. \matlab{} regroupe à la fois un langage de script, Un environnement de développement ainsi que des modules -- appelées toolbox -- offrant des fonctionnalités de haut niveau spécialisés dans divers domaines scientifique.

La toolbox \emph{Symbolic Math}\footnote{Page web de la \href{http://fr.mathworks.com/products/symbolic/}{Symbolic Math Toolbox}.}, permet notamment de résoudre des systèmes d'équations (grâce à des solveurs\footnote{Les différents solveurs proposés par le toolbox : \url{http://web.archive.org/web/20150602192607/http://fr.mathworks.com/help/symbolic/algebraic-equations-and-systems.html?nocookie=true}}) ce qui est intéressant car les principales contraintes que je serait amener à résoudre se ramèneront à des systèmes d'équations.

\matlab{} à pour avantage d'être une solution clé en main permettant d'avoir des résultat rapidement, je m'en sert donc pour tester mes idées avant d'aller plus loin dans l'implémentation.

\subsubsection{\gecode{}}

\gecode{} est la bibliothèque logicielle actuellement utilisée par la dernière release d'\iscore{}\footnote{Lien Github du module CSP d'\iscore{} \url{https://github.com/i-score/JamomaCore/tree/dev/Score/extensions/Scenario/source}}\footnote{En annexe \ref{HeritageI-scoreNew}}. 
Gecode existe depuis 2005\footnote{La page de changeLog de \gecode{} : \url{http://web.archive.org/web/20131122052359/http://www.gecode.org/doc-latest/reference/PageChange.html}}, la bibliothèque est codée en \cpp{} et sa rapidité\cite{Minizinc2015} ainsi que sa license permissive (MIT\cite{Mit2015}) en on fait une solution utilisée par une grande communauté.

\subsubsection{\cassowary{}}

\begin{citeauteur}[Greg J. Badros\footnotemark{}]
Cassowary is an incremental constraint solving toolkit that efficiently solves systems of linear equalities and inequalities. Constraints may be either requirements or preferences. Client code specifies the constraints to be maintained, and the solver updates the constrained variables to have values that satisfy the constraints.
\end{citeauteur}

Cassowary est à la fois un algorithme\cite{Borning:1997:SLA:263407.263518} et un solveur de contraintes\cite{Badros98thecassowary} écrit en Smalltalk, \cpp{}, Java, and Python\footnotemark[\value{footnote}]{}.

\footnotetext{Page web de \cassowary{} : \url{http://web.archive.org/web/20150603075607/http://constraints.cs.washington.edu/cassowary/}}

Tel que décrit dans l'article\cite{Borning:1997:SLA:263407.263518}, \cassowary{} a été conçu pour travailler avec des interfaces graphiques, l'algorithme est notamment utilisé par l'auto layout de Apple\cite{sadun2013ios} qui gère le positionnement des éléments graphiques grâce à des contraintes que spécifie l'utilisateur.

Il existe des implémentations modernes de cassowary, notamment une en \cpp{} appellée \kiwi{}\footnote{Github du projet \kiwi{} : \url{https://github.com/nucleic/kiwi}}

\subsubsection{\ortools{}}
















\section{\iscore{}}

Le logiciel \iscore{} présente une implémentation du concept de scénarios interactifs, nous allons voir comment ce projet est né, quelles sont ses ambitions et surtout quelle est sont utilisation des solveurs de contraintes. Si, comme nous allons le voir, les solveurs ont été utilisés dès le début comme une aide à l'édition de pièces musicales -- comme c'est le cas au sein du logiciel Boxes -- ils se sont avérés très utiles lorsque que la notion d'interactivité est apparu avec \iscore{}.

\subsection{boxes}% (origine etc)

Boxes\cite{Beurive2000boxes} est un logiciel de composition musicale qui propose un plan de travail en deux dimensions \figureRef{boxesPreview} sur lequel on vient placer des portions audio contenues dans des boites. Ces boites peuvent elles-mêmes êtres contenues dans d'autres de façon à créer une hiérachie.

\schema{boxesPreview}{L'interface de \boxes{}}

Le fonctionnement est assez simple, c'est la position sur l'axe horizontal qui détermine le moment ou sera joué une boite. Le positionnement des boites étant directement lié à la structure temporelle du morceau, un système de contraintes a été mis en place pour protéger l'intégrité de la pièce lors des modifications du compositeur.

Les contraintes de boxes découlent de relations d'Allen \figureRef{allen} que le compositeur peut spécifier entre chacune des boites pour en conserver l'agencement.

\schema{allen}{Les différentes relations de Allen illustrées.}

Une des premières implémentations d'\iscore{} s'est appuyé sur \openmusic{}, un logiciel utilisé pour la composition musicale. \openmusic{} possède un environnement de "maquette" qui ressemble à \boxes{}. \openmusic{} se démarque par le fait qu'il soit orienté pour faire de la synthèse audio grâce à la technologie \osc{} et grace à la notion d'évènements qui lui permettent d'envoyer des commandes à des modules externes comme des patchs Max ou Pure Data. \openmusic{} offre également la possibilité de déclencher manuellement l'envoi de messages \osc{} ce qui introduit déjà de l'interactivité.

Les chercheurs du \acrshort{scrime}\cite{Scrime2015} ont souhaités intégrer les contraintes présentes \boxes{}\cite{allombert:hal-00353628} pour permettre de contrôler l'interactivité et de l'inscrire dans la démarche de composition.

Ce faisant, les utilisations ont mis en avant le besoin d'introduire des contraintes plus fines que les relations d'Allen. Ces nouvelles contraintes -- que nous appellerons relations temporelles -- contraignent deux évènements entre eux par un intervalle de durée $[Val_{min}, Val_{max}]$ les séparant. Cet intervalle est lui-même inclus dans l'intervalle $[0,+\infty]$, et on en distinguera trois types :

\begin{itemize}
\item rigide, quand $Val_{min} = Val_{max}$ ;
\item souple, quand $[Val_{min}, Val_{max}] = [0,+\infty]$ ;
\item semi-rigide, pour tout autre $[Val_{min}, Val_{max}]$.
\end{itemize}

Ces relations temporelles vont alors être utilisés dès la version 0.2 d'\iscore{}.

\subsection{\iscore{} 0.2}

La version 0.2 d'\iscore{}\footnote{\url{https://github.com/i-score/i-score/tree/release/0.2}} arrive avec un formalisme détaillé\cite{hogue2014ossia} décrivant les éléments destinés à composer les scénario interactifs ainsi que les principes desquels ils découlent. On peut voir \figureRef{iscore02} dans cette version l'héritage de \boxes{} et \openmusic{} par l'utilisation de boites et d'évènements.

\schema{iscore02}{L'interface d'\iscore{} 0.2 : l'explorateur de contrôles \osc{} à gauche et la surface d'édition à droite. On peut voir une relation temporelle semi-rigide entre Box2 et Box3 et une relation rigide entre Box2 et Box4. La durée de la boite Box4 est une relation souple. Chaque relation non rigide entraîne le création d'un trigger point -- ou point de déclenchement --, une condition binaire qui permet par exemple écouter la valeur d'une variable. }

Les relations temporelles induisent des contraintes qui sont gérées avec l'aide d'un solveur de contrainte, actuellement \gecode{} 

%moteur d'exécution

\subsection{\iscore{} 0.3}% (édition & execution)

La dernière version d'\iscore{}, actuellement en développement\footnote{\url{https://github.com/OSSIA/i-score}} \iscore{}, est la version 0.3.

Cette nouvelle version a plusieurs objectifs :

\begin{itemize}
    \item améliorer les contrôles et l'intuitivité de l'interface ;
    \item pouvoir inclure des scénarios dans les boites d'autres scénarios ;
    \item intégrer les boucles\footnote{Une boucle est le fait de répéter une boite.} ;
    \item donner plus d'importance aux relations temporelles ;
    \item pouvoir programmer des scénario sur des FPGA\footnote{\emph{field-programmable gate array} ou \emph{Circuit logique programmable}, ce sont des circuit intégré construit à base de portes logiques. Ils ont l'avantage de présenter une fiabilité et une stabilité supérieure à un ordinateur personnel.} ;
    \item lors de l'exécution :
    \begin{itemize}
        \item n'afficher qu'un seul curseur de temps ;
        \item afficher les évènements passé à la position exacte à laquelle ceux-ci se sont déroulés.
    \end{itemize}
\end{itemize}

Pour l'instant, cette version ne présente qu'une interface d'édition et utilise le moteur de la version 0.2 pour exécuter des scénarios. Le moteur d'exécution d'\iscore{} 0.3 est développé par Jaime Arias, grâce à l'utilisation d'automates

elle n'intègre pas encore de solveur de contraintes et le positionnement des éléments est pour l'instant géré de façon empirique.

